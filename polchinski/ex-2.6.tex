%\documentclass[aps,prd,preprint]{revtex4-1}
\documentclass[aps,prd,final,twocolumn,floats,floatfix,nofootinbib,10pt]{revtex4-1}

% Note:  comment out one of the two documentclass commands, depending
% on whether you need the final or preprint version. It is most convenient
% to work in preprint mode, and switch to final only at the end.

\usepackage{graphicx}
\usepackage{amsmath}
\usepackage{amsfonts}
\usepackage{amssymb}
\usepackage{amstext}
\usepackage[english]{babel}
\usepackage{helvet}
\usepackage{microtype}
\usepackage{dsfont}
\usepackage[pdftex]{hyperref}
\usepackage{tikz}

\special{papersize=8.5in,11in}
\setlength{\pdfpageheight}{\paperheight}
\setlength{\pdfpagewidth}{\paperwidth}

\begin{document}

\title{Exercise (2.6) in Polchinski}
\author{indutny}
\date{\today}
\noaffiliation

\maketitle

\section{Problem statement}

Given the transformation law for the metric:
\begin{align}
  \delta g_{ab} = -\partial_a v_b - \partial_b v_a
\end{align}
determine the most general $v^a(\sigma)$ that leaves flat $d$-dimensional
Euclidian metric $\delta_{ab}$ invariant up to a local rescaling.

\section{Solution}

The case of $d = 2$ is covered extensively in String Theory textbooks, so we
will concentrate our efforts on $d > 2$ here.

\subsection{Notation}

Everywhere below $a$, $b$, and $c$ are three different indices:
$a \neq b$, $b \neq c$, $a \neq c$. Note that it means that most of the
derivation does not apply to $d = 2$. This the reason why $d = 2$ is special.

$m$ and $n$ are possibly equal, but both non-negative integers: $m \geq 0$,
$n \geq 0$.

\subsection{Derivation}

The local rescaling invariance imposes following constraints:
\begin{align}
  \partial_a v_a & = \partial_b v_b, \label{eq:diagonal} \\
  \partial_b v_a & = -\partial_a v_b \label{eq:off-diagonal}.
\end{align}

For $d > 2$ we can derive an additional constraint:
\begin{align}
  \partial_c \left( \partial_b v_a \right) & =
  -\partial_c \left( \partial_a v_b \right) =
  -\partial_a \left( \partial_c v_b \right) = \notag \\
  \partial_a \partial_b v_c & = \partial_b \left( \partial_a v_c \right),
\end{align}
which is equivalent to:
\begin{align} \label{eq:two-vars}
  \partial_b \left( \partial_c v_a - \partial_a v_c \right) =
  2 \partial_b \partial_c v_a = 0.
\end{align}

This immediately implies that $v_a$ can't have terms with three different
coordinate variables:
\begin{align}
  v_a & = \sum_{\overset{m \geq 0, n \geq 0}{b \neq a}}
  c^{(ab)}_{mn} (\sigma_a)^m (\sigma_b)^n, \\
  c^{(ab)}_{m0} & = c^{(ac)}_{m0}.
\end{align}

In fact, the space of solutions is even smaller. The term
\begin{align}
  v_a = (\sigma_a)^{m + 2} (\sigma_b)^{n + 1} + \cdots
\end{align}
is incompatible with constraints:
\begin{align}
  \partial_c v_c = \partial_a v_a =
    (m + 2) (\sigma_a)^{m + 1} (\sigma_b)^{n + 1},
\end{align}
but $v_c$ can't have a term proportional to
$(\sigma_c) (\sigma_a)^{m + 1} (\sigma_b)^{n + 1}$ due to \eqref{eq:two-vars}.
Therefore
\begin{align} \label{eq:corner}
  c^{(ab)}_{(m + 2)(n + 1)} = 0.
\end{align}
As we shall see only finitely few of them are non-zero, and even less
are independent of each other.

The first constraint \eqref{eq:diagonal} can be translated from differential
equation
\begin{align}
  \partial_a v_a & = \sum m c^{(ab)}_{mn} (\sigma_a)^{m - 1} (\sigma_b)^n, \\
  \partial_b v_b & = \sum n c^{(ba)}_{nm} (\sigma_b)^{n - 1} (\sigma_a)^m,
  \partial_a v_a = \partial_b v_b
\end{align}
to a relation between coefficients
\begin{align} \label{eq:first-rel}
  (m + 1) c^{(ab)}_{(m + 1) n} = (n+1) c^{(ba)}_{(n + 1) m}.
\end{align}

Similarly \eqref{eq:off-diagonal} translates to
\begin{align} \label{eq:second-rel}
  (n + 1) c^{(ab)}_{m (n + 1)} = -(m + 1) c^{(ba)}_{n (m + 1)}.
\end{align}

Combining \eqref{eq:first-rel} and \eqref{eq:second-rel} together we see that
\begin{align} \label{eq:third-rel}
  c^{(ab)}_{(m + 2) n} = -\frac{(n + 2) (n + 1)}{(m + 2) (m + 1)}
    c^{(ab)}_{m (n + 2)},
\end{align}
which together with \eqref{eq:corner} implies that
\begin{align}
  c^{(ab)}_{m (n + 3)} \simeq c^{(ab)}_{(m + 2) (n + 1)} = 0, \\
  c^{(ab)}_{(m + 4) n} \simeq c^{(ab)}_{(m + 2) (n + 2)} = 0.
\end{align}
Therefore the only coefficients that can be non-zero are
\begin{align}
  c^{(a)} \equiv c^{(ab)}_{00}, \; c^{(a)}_{1} \equiv c^{(ab)}_{10}, \;
  c^{(ab)}_{01}, \\
  c^{(ab)}_{11}, \; c^{(ab)}_{12}, \; c^{(a)}_{3} \equiv c^{(ab)}_{30}.
\end{align}
We can also work out relations between them. Using \eqref{eq:first-rel} and
\eqref{eq:second-rel}:
\begin{align}
  c^{(a)}_{1} & = c^{(ab)}_{10} = c^{(ba)}_{10} = c^{(b)}_{1},
    \label{eq:first-family} \\
  c^{(ab)}_{01} & = -c^{(ba)}_{01} \label{eq:second-family}, \\
  c^{(ab)}_{11} & = -2 c^{(ba)}_{02} = 2 c^{(ba)}_{20} = 2 c^{(b)}_{2},
    \label{eq:third-family} \\
  c^{(a)}_{3} & = c^{(ab)}_{30} = \frac{1}{3} c^{(ba)}_{12} =
    -\frac{1}{3} c^{(ab)}_{12} = -c^{(ba)}_{30} = -c^{(b)}_3{}
    \label{eq:fourth-family},
\end{align}
but the \eqref{eq:fourth-family} has to be zero
\begin{align}
  c^{(a)}_{3} = -c^{(b)}_{3} = c^{(c)}_{3} = -c^{(a)}_3 = 0.
\end{align}

Before writing out the most general form of $v^a$, let us count the number of
independent solutions
\begin{align}
  \# \left\{ c^{(a)} \right\} & = d, \\
  \# \left\{ c^{(a)}_1 = c^{(b)}_1 \right\} & = 1, \\
  \# \left\{ c^{(ab)}_{01} = - c^{(ba)}_{01} \right\} & =
    \frac{d (d - 1)}{2}, \\
  \# \left\{ c^{(a)}_{2} \right\} & = d,
\end{align}
and the total
\begin{align}
  & d + 1 + \frac{d (d - 1)}{2} + d \\
  & = \frac{2 d + 2 + d^2 - d + 2 d}{2} \\
  & = \frac{d^2 + 3 d + 2}{2} \\
  & = \frac{(d + 1) (d + 2)}{2}.
\end{align}

\subsection{Result}

The most general form of solution is:
\begin{align}
  v^a = & c_0 + c_1 (\sigma_a) + c_2 (\sigma_a)^2 \notag \\
        & + \sum_b \left[ c_3 (\sigma_a) (\sigma_b) + -2 c_4 (\sigma_b)^2
          \right] \notag \\
        & + \sum_b c_5 \left[
          (\sigma_a)^3 - \frac{1}{3} (\sigma_a) (\sigma_b)^2 \right].
\end{align}

\end{document}
