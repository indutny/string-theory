%\documentclass[aps,prd,preprint]{revtex4-1}
\documentclass[aps,prd,final,twocolumn,floats,floatfix,nofootinbib,10pt]{revtex4-1}

% Note:  comment out one of the two documentclass commands, depending
% on whether you need the final or preprint version. It is most convenient
% to work in preprint mode, and switch to final only at the end.

\usepackage{graphicx}
\usepackage{amsmath}
\usepackage{amsfonts}
\usepackage{amssymb}
\usepackage{amstext}
\usepackage[english]{babel}
\usepackage{helvet}
\usepackage{microtype}
\usepackage{dsfont}
\usepackage[pdftex]{hyperref}
\usepackage{tikz}

\special{papersize=8.5in,11in}
\setlength{\pdfpageheight}{\paperheight}
\setlength{\pdfpagewidth}{\paperwidth}

\begin{document}

\title{Deriving (2.8.25a) in Polchinski}
\author{indutny}
\date{\today}
\noaffiliation

\maketitle

\section{Target formula}

\begin{align}\label{eq:25a}
\alpha_n = - \frac{i n}{(2 \alpha')^{1/2}} X_{-n} - i \left( \frac{\alpha'}{2} \right)^{1/2}
  \frac{\partial}{\partial X_n} \tag{2.8.25a}
\end{align}

\section{Derivation}

Start by noting that:
\begin{align}
X & = X_i + X_0 +
  \sum_{n=1}^\infty \left( z^n X_n + \overline{z}^n X_{-n} \right), \\
S & = \frac{1}{2 \pi \alpha'} \int_{|z| < 1} d^2z \;
  \partial X \bar{\partial} X.
\end{align}

Computing a partial derivative of the action with respect to the boundary field
mode we get:
\begin{align}
\frac{\partial S}{\partial X_n} & = \frac{1}{2 \pi \alpha'} \int_{|z| < 1} d^2z
  \; \partial (z^n) \bar{\partial} X \\
  & = \frac{1}{2 \pi \alpha'} \int_{|z| < 1} d^2z \;
    \partial( z^n \bar{\partial} X) \\
  & = i \frac{1}{2 \pi \alpha'} \oint_{|z| = 1} d\bar{z} \;
    z^n \bar{\partial} X \\
  & = i \frac{1}{2 \pi \alpha'} \oint_{|z| = 1} d\bar{z} \;
    \bar{z}^{-n} \bar{\partial} X \\
  & = \frac{1}{\alpha'} n X_{-n}.
\end{align}

Which means that the wavefunctional is an eigenstate of such operator:
\begin{align}
\frac{\partial}{\partial X_n} \Psi_1[X_b] & =
  \frac{\partial}{\partial X_n} \int [dX_i]_{X_b = 0} \exp(-S) \\
  & = -\frac{n}{\alpha'} X_{-n} \left( \int [dX_i]_{X_b = 0} \exp(-S) \right) \\
  & = -\frac{n}{\alpha'} X_{-n} \Psi_1[X_b]
\end{align}
or, using the factors from \ref{eq:25a} we see that the first and the second
terms of \ref{eq:25a} cancel each other when applied to $\Psi_1$
\begin{align}
-i \left(\frac{\alpha'}{2}\right)^{1/2} \frac{\partial}{\partial X_n}
  \Psi_1[X_b] = -\frac{in}{(2 \alpha')^{1/2}}  X_n \Psi_1[X_b].
\end{align}

Now using the definition of $\alpha_m$ we can compute it's value using the
boundary contour
\begin{align}
  \alpha_m = i \left(\frac{2}{\alpha'} \right)^{1/2} \oint_z
  \frac{dz}{2 \pi i} z^m \partial X_b(z) =
  -in \left(\frac{2}{\alpha'} \right)^{1/2} X_{-n}
\end{align}
which is precisely twice the first term of \ref{eq:25a}.

I have proven nothing.

\end{document}
